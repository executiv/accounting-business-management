\section{July 16, 2014}

\subsection{Performance Indicators}
\begin{outline}
\1 Performance indicator is the name used for the way a business checks how well they are following their business objectives.
\2 Profitability
\3 Profitability is usually measured by calculating either the dollar profit of the business, or the gross profit ratio, which outlines the price they buy stock at, compared with the price they sell stock at.
\2 Employee satisfaction
\3 Employee satisfaction is measured by one of three things: staff turnover, number of employee complaints, and the results of employee satisfaction surveys. Often the results would be generally more accurate if the surveys are anonymous as opposed to by name, as people are more likely to write what they actually think when there are no potential repercussions.
\2 Quality of product or service
\3 The quality of a product or service can be measured through a number of methods. The business can ask customers for testimonials or to fill out feedback forms, which in turn can give the business an idea of the quality of their product. Safety standards can also be used as a benchmark for products or services.
\2 Customer service/satisfaction
\3 Customer service quality and customer satisfaction can be measured with testimonials, the number of repeat customers, or the results of customer satisfaction surveys.
\2 Market recognition
\3 Market recognition can be identified by calculating market share, and looking at the number of new customers.
\2 Environmentally friendly
\3 The degree to which a business is environmentally friendly is easily measured, by looking at the results of pollution audits.
\2 Occupational Health and Safety
\3 Whether a business is following occupational health and safety guidelines properly can easily be found by looking at the number of workplace accidents and injuries.
\end{outline}

\subsection{Industries}
\begin{outline}
\1 The industry of a business refers to the type of business activity that a business undertakes.
\2 Examples
\3 Agriculture, Forestry and Fishing
\3 Mining
\3 Manufacturing
\3 Electricity, Gas and Water Supply
\3 Construction
\3 Wholesale Trade
\3 Retail Trade
\3 Accommodation, Cafes and Restaurants
\3 Transport and Storage
\3 Communication Services
\3 Finance and Insurance
\3 Property and Business Services
\3 Education
\3 Health and Community Services
\3 Cultural and Recreational Services
\3 Personal and Other Services
\1 Different industries have different business sizes, which influences the distribution of an industry's small, medium and large businesses.
\2 In the small business sector, the industries with the largest number of businesses are, respectively:
\3 Property and Business Services
\3 Construction
\3 Retail Trade
\2 In the medium business sector, the industries with the largest number of businesses are, respectively:
\3 Property and Business Services
\3 Retail Trade
\3 Manufacturing
\2 In the large business sector, the industries with the largest number of businesses are, respectively:
\3 Property and Business Services
\3 Manufacturing
\3 Retail Trade
\end{outline}