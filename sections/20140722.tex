\section{July 22, 2014}

\subsection{Ownership Structures}
\begin{outline}
\1 Private company
\2 A private company can be set up by 1 to 50 shareholders. Companies need to be registered with the Australian Securities and Investment Commission (ASIC) and must adhere to laws that govern the operation of companies. Private companies are seperate legal entities, where the company pays tax, rather than the owners, and can distribute profits to shareholders through dividends. Shares in companies can be sold with the consent of the owner.
\2 Advantages
\3 Limited liability means the personal assets of shareholders are protected if the business accrues debt.
\3 As the company is a separate entity it can be bought/sold.
\3 Advantageous for businesses making large amounts of money.
\3 Owner or owners can raise capital through selling shares
\2 Disadvantages
\3 Setup and registration fees are costly for 
\3 Can lead to a loss of control for the owner.
\3 Tax rate of 30\% is higher than the lower rates for personal income tax, so companies earning low profits pay more tax than other ownership structures.
\end{outline}

\subsection{Small Business}
\begin{outline}
\1 Failure of small business
\2 Statistically, most small businesses fail within the first 5 years of operation.
\3 Most common causes
\4 Inefficient processes
\4 Selling products or services at too low a price.
\4 Business taken by a much larger businesses within the same industry
\4 Being sued by another individual or business.
\4 Inadequate start-up capital
\4 Failure to repay interest in borrowings
\4 Inadequate advertising and promotion
\4 Can't absorb losses from customers failing to pay
\4 Lack of management experience or poor decision-making
\4 Lack of financial knowledge or record keeping
\4 Lack of formalised business plan
\4 Business idea not unique enough
\4 Poor choice of location
\end{outline}

\subsection{Business Plan}
\begin{outline}
\1 Executive summary
\2 Will include the rationale for the business being successful
\2 Includes an overview of each of the major sections -- operations, marketing, and finance
\2 1 page long
\1 Operations
\2 Explain how the business will operate on a daily basis.
\3 Hours of operation
\3 Staffing
\4 Number of staff
\4 Responsibilities
\3 Facilities
\3 Location
\1 Marketing
\2 Explain how the business would handle its relationship with customers
\3 Identify the target market
\3 Marketing strategies
\1 Finance
\2 Explain how much finance is needed
\3 What would the money be used for
\3 Where is it coming from
\3 Demonstrate how the business will make money using a cash flow statement
\end{outline}
